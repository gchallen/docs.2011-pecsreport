\textbf{\scshape Leads:} Andrew Campbell, Geoffrey Challen

\textbf{\scshape Participants:} (1)~Gaetano Borriello, (2)~Andrew Campbell,
(3)~Roy Campbell, (4)~Geoff Challen, (5)~David Chu, (6)~Sajal Das, (7)~Mario
Di Francesco, (8)~Mads Haahr, (9)~Ahmed Helmy, (10)~David Kotz, (11)
Narayanan Krishnan, (12)~Mohan Kumar, (13)~Thomas Little, (14)~Jie Liu, (15)
Mirco Musolesi, (16)~Kishore Ramachandran, (17)~Mahadev Satyanarayanan, (18)
Andreas Savvides, (19)~Bill Schilit, (20)~David Wetherall, (21)~Vincent Wong,
(22)~Feng Zhao, (23)~Gang Zhou.

\section{Introduction}

Phones are the first pervasive mobile computing technology. Between 1990 and
2010 the number of mobile phone subscriptions grew by two orders of
magnitude. Today's phones are migrating from so-called feature
phones---limited to voice and text messaging---to smartphones which integrate
powerful processors, multiple communication technologies, ample storage and
sensor suites. \uline{The ubiquity and increasing capabilities of smartphone
devices make them our best option for realizing the pervasive computing
vision at scale.}

\section{Current Status}

Today's smartphone is as powerful as larger mobile devices were several years
ago. It integrates multiple processors, including some specialized for
specific tasks. It can communicate data over 1,000s of meters to cellular
towers using 3G or 4G, over 10s of meters to 802.11 access points using
Wi-Fi, and over 1s of meters to many other devices using Bluetooth. This
array of communication technologies mean that phones may provide last-hop
communication to body area and other deployed sensors that lack the power
required for long-distance communication. Cheap and plentiful storage allows
smartphones to cache a great deal of information, and the growing power of
the cloud allows them to offload expensive computation. The emergence of
application distribution channels like the Apple AppStore and Google Android
Market have accelerated smartphone innovation by providing access to millions
of deployed iPhone and Android devices.

\section{Vision and Challenges}

To frame our discussion of the future of smartphone research, our session
outlined a vision of the smartphone in 2020. We imagine the capabilities of
Phone~2020 and some exciting future applications below. Working backward, we
develop a set of research challenges that must be addressed before Phone~2020
can become reality.

\subsection{Phone 2020: The Next Decade of the Smartphone}

In a distracted world, Phone~2020 will help us deal with the data deluge by
offloading much of the current human burden caused by information overload.
Phone~2020 is itself continuously capturing large quantities of data about
our lives---including location traces, readings from internal and external
sensors, and logs of our mobile-based activities---and contributing to the
steady increase in data collection. But it will also help analyze and
interpret these new data streams to maximize their value. By learning our
patterns, Phone~2020 will make suggestions about our daily lives, anticipate
our actions, and become woven into the fabric of our existence.
 
In order to assist us, the future smartphone will interact with
everything---other phones, the cloud, nearby sensors and actuators, vehicles
and buildings---and display information in ways tailored to each user. It
will process the environment and help us discover and navigate the world
around us, including visibility into social networks. By better understanding
users, the Phone~2020 will manage their attention and know when to interrupt.
Through an increase in its own capabilities and by seamlessly inter-operating
with powerful cloud resources, Phone~2020 will be starting to make desktop
and laptop computers obsolete.
 
The new capabilities of Phone~2020 will support new applications that open up
new markets. Smartphones will define the classroom of the future. They will
augment reality to further education, socialization, health care and gaming.
They will sense reality to manage cities, workplaces and traffic while
continuously recording our digital lives. Smartphones of the future will help
us work more efficiently, serving as portable office and personal digital
assistant, conserving useful working hours and creating time for leisure and
entertainment. We expect future applications to be long-lived---leveraging
continued interaction with users over time---and local---exploiting the
density of smartphone penetration to augment or replace communications
infrastructure, critical in developing countries where such infrastructure
may be unreliable or nonexistent. The ubiquity of smartphones and their
proximity to their human users will make them a critical component of future
approaches to disaster relief and emergency management.
 
Phones will also continue to be integrated with online social networks.
Smartphones are already the quintessential social device. The desire of
people to connect with each other drove the adoption of cellular phone
technologies. With social networking exploding on the Internet in 2011,
Phone~2020 unites the social network with the social device. It will help us
further understand the structure of existing social structures, while
assisting in the formation of ad-hoc social networks grounded in physical
gatherings of people with similar interests. Phone~2020 will also contribute
to network science by monitoring user behavior and supporting applications
such as disease tracking.

\subsection{Challenges}

In order to build Phone~2020, we identified a number of challenges that our
community must address. These divide into three categories: (1) developing
the capabilities of the smartphone and its environment, (2) improving
interaction between smartphones and users, and (3) coping with the potential
for massive large-scale data collection using smartphone-integrated sensors.
 
The Phone~2020 vision is predicated on continued improvements to smartphone
and smartphone infrastructure performance. Future smartphones must be more
powerful, communicate more quickly, store more data, and integrate new
interaction technologies. Unfortunately, these goals are at odds with data
bandwidth and battery capacities, both of which are scaling slowly. We expect
future smartphones to deploy opportunistic algorithms that multiplex both
time and space in order to improve performance. The overall heterogeneity of
deployed devices and standards is another challenge limiting device-to-device
inter-operation and the potential for Phone~2020 to interact with all the
devices it encounters. We also discussed the importance of integrating the
smartphone with existing Wi-Fi networks to improve connectivity and network
performance. Peer-to-peer architectures were suggested as a potential way to
improve performance, particularly when infrastructure is lacking.
 
Another property that is not scaling is human attention. We already pay too
much attention to our smartphones to believe that we have achieved the
invisibility captured by early visions of ubiquitous computing, and this
problem is worsening. To better optimize our attention future smartphones
must deploy interfaces allowing more nuanced interaction with users and
capable of processing emotional cues. To improve the interaction between
humans and their devices, new algorithms must be developed enabling
behavior-based modeling, computing, and testing. In addition, user interfaces
need to be reconsidered, including those that, while unsuitable for larger
devices, may work well on smartphones. Phone~2020, with its ability to
interact seamlessly with objects around it, will be able to leverage
``found'' interface elements in the environment to enable much richer
interaction modalities than those possible on the smartphone itself.
 
Smartphones hold the potential both to contribute to and to alleviate the
growing data deluge. Large-scale deployment of sensor suites on smartphones
combined with cheap bandwidth and storage will lead to a growing amount of
data produced by the smartphones of the future. Securing this
information---much of it sensitive and personal---will be a major challenge.
Designed as a personal device, smartphones are increasingly interacting with
each other and the environment, creating new opportunities to steal and
misuse information. Developing security and privacy models that users can
understand and adapt to their needs is a critical challenge to the continued
advance of this technology.
 
Interpreting and processing the collected data will also be difficult. There
are opportunities for harnessing the distributed power of large numbers of
smartphones through collaborative computation. These capabilities, if
developed, might complement the continued aggregation of computation in the
cloud. Fundamentally, however, the smartphone of the future will be a portal
to the intelligent processing and management of data in order to reduce user
distraction and allow users to focus their attention elsewhere.

\section{Recommendations to NSF}

Our recommendations highlight areas where the research community can make
significant and distinct contributions. Industry is already very active in
this space and has many advantages, particularly when working at scale.
However, there remain many opportunities for the academic community to
develop the future smartphone in directions complementary to those being
pursued by industry.

\subsection{Research Support}

We recommend that the NSF develop research programs addressing the key
challenges to realizing the Phone 2020 vision outlined above:

\begin{enumerate}

\item We need to continue the development of smartphone and infrastructure
capabilities to support demanding new applications.

\item We must tear down the walls that divide devices from each other and
limit the ability of the smartphone to fully understand its environment.

\item We need better interfaces allowing the future smartphone to conserve human
attention.

\item We need smartphones to help users cope with the ever increasing amount of
data accessible to and collected about them.

\item We need security and privacy models that users can understand and adapt
to match their expectations and the current context---the highly dynamic pool
of surrounding devices and communications channels, the social setting, and
the user’s activity. 

\item We believe it is important to understand and document our continued
co-evolution with our mobile devices: how we are changing them, how they are
changing us.

\end{enumerate}

\subsection{Infrastructure Support}

To enable academics to succeed at complementing industry, the NSF should
provide them with resources and infrastructure facilitating experimentation
at scale. NSF can also take a role in partnering with industry to gain access
to large numbers of smartphones, air time, call logs or other large data
sets. Further partnerships with industry might also allow us to do
citizen-driven science in other areas that leverage the smartphone as a
pervasive computing platform.

Application distribution channels like the Android Market and Google AppStore
also provide academics with the opportunity to deploy research systems at
scale by leveraging channels established by industry. We can release our own
code on the AppStore, perhaps piggybacking on top of other popular
applications. Users worldwide might be willing to participate in a
large-scale virtual laboratory. At sufficient scale such a laboratory could
provide built-in guarantees to researchers allowing academic research to
reach large numbers of deployed smartphones.

\subsection{Educational Opportunities}

One germane direction for the academic community to explore is in the use of
smartphones to enhance education and learning. The future smartphone may
enter the classroom and help put lessons in context, as well as extending the
reach of learning beyond the classroom.
 
We also believe that continued growth and competitiveness in the smartphone
market depends on educating the next generation of computer scientists on
smartphone development. Given the centrality of the smartphone and the cloud
to future computing, we must train engineers that can help integrate these
two technologies in ways that harness the properties and capabilities of
both. We recommend support for the continued development of courses in
smartphone programming, application development, and smartphone-cloud
interaction.
