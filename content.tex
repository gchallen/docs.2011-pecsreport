\textbf{\scshape Leads:} Andrew Campbell, Geoffrey Challen

\textbf{\scshape Participants:} Gaetano Borriello, Andrew Campbell, Roy
Campbell, Geoff Challen, David Chu, Karthik Dantu, Sajal Das, Mario Di
Francesco, Prabal Dutta, Mads Haahr, Ahmed Helmy, David Kotz, Narayanan
Krishnan, Mohan Kumar, Thomas Little, Jie Liu, Mirco Musolesi, Kishore
Ramachandran, Mahadev Satyanarayanan, Andreas Savvides, Bill Schilit, David
Wetherall, Vincent Wong, Feng Zhao, Gang Zhou

\section{Introduction}

Phones are the first truly pervasive mobile computing technology. Between
1990 and 2010 the number of mobile phones subscriptions grew by two orders of
magnitude. Today’s phones are migrating from so-called feature phones limited
to voice and text messaging to smartphones, integrating powerful processors,
multiple communication technologies, ample storage and sensor suites. Phones
are where people are and the capabilities of emerging smartphone devices
cause them to be our current best option for implementing the pervasive
computing vision at truly global scale.

\section{Current Status}

Today’s smartphone is as powerful as larger mobile devices---such as
laptops---were only a few years ago. It features multiple processors, including
many specialized for specific tasks. It can communicate data to cellular
towers using 3G and 4G, 802.11 access points using Wi-Fi, and a variety of
other devices over Bluetooth. Increasingly cell phones are providing last-hop
communication to body area and other deployed sensors. Cheap and plentiful
storage allows phones to cache a great deal of information, and the growing
power of the cloud allows them to potentially offload expensive or
time-consuming computation. The emergence of application distribution
channels like the Apple AppStore and Google Android Market have accelerated
innovation in the smartphone space by providing developers easy access to the
millions of deployed iPhone and Android devices.

\section{Vision and Challenges}

To frame our discussion of the future of smartphone research, our session
outlined a vision of the phone of the future---Phone 2020. We describe the
capabilities of this new platform and some applications to expect to move to
or be developed for this phone of the future below. Working backward, we
develop a set of research challenges currently limiting our progress towards
Phone 2020.

\subsection{Phone 2020: The Next Decade of the Smartphone}

In an increasingly distracted world filled with information, Phone 2020 will
help us deal with this data deluge by offloading much of the current human
burden caused by information overload. Despite the fact that Phone 2020 is
itself continuously capturing large quantities of data about our
lives---including location traces, readings from internal and external sensors,
and logs of our mobile-based activities---and contributing to the steady
increase in data collection, it will also help analyze and interpret these
new data streams to maximize their value. By learning our patterns, Phone
2020 will be able to make suggestions about our daily lives, anticipate our
actions, and will become increasingly woven into the fabric of our existence
and involved in a variety of new undertakings.
 
In order to assist us, the future phone will interact with everything---other
phones, the cloud, nearby sensors and actuators, vehicles and buildings---and
display information in ways that suit each user. It will process the
environment and proactively help us discover and navigate the world around
us, including visibility into social networks. By better understanding its
user, the Phone 2020 will manage their attention and know when to interrupt.
Through an increase in its own capabilities and by seamlessly inter-operating
with powerful cloud resources, Phone 2020 will be on the way to making
desktop and laptop computers obsolete.
 
The new capabilities of Phone 2020 will drive exciting new applications and
open up new markets. Smartphones will enter and define the classroom of the
future. They will augment reality to further education, socialization, health
care and gaming. They will sense reality to manage cities, workplaces and
traffic while continuously recording our digital lives. Smartphones of the
future will help us work more efficiently by constituting a real mobile
office and personal digital assistant, conserving useful working hours and
creating time for leisure and entertainment. We expect future applications to
be long-lived---leveraging continued interaction with users over time---and
local---perhaps exploiting the density of smartphone penetration to augment
or replace communications infrastructure, critical in developing countries
where this infrastructure may be nonexistent or unreliable. The sheer number
of smartphones and their colocation with their human users may make them a
critical component of future approaches to disaster relief and emergency
management.
 
We also foresee the continued integration of phones with online social
networks. Smartphones are already the quintessential social device. From
voice to electronic communication, the desire of people to connect with each
other drove the initial adoption of cellular phone technologies to the scales
currently achieved. Now that social networking has taken off on the Internet,
we see opportunities to further unite the social network with the social
device. This can help us further understand the structure of existing social
networks, while phones also assist in the formation of ad-hoc social networks
grounded in physical gatherings of people with similar interests. In the
context of social networks we should also not ignore the potential for phones
to contribute to network science by monitoring user behavior and assisting in
social applications like tracking disease propagation.

\subsection{Challenges}

In order to build Phone 2020, we identified a number of challenges that our
community must address. These roughly divide into three categories:
developing the capabilities of phone and their environment, improving
interaction between phones and users, and coping with the potential for
massive large-scale data collection using phones as sensors.
 
The Phone 2020 vision is predicated on continued improvements to smartphone
and smartphone infrastructure performance. Future smartphones must be more
powerful, communicate more quickly, store more data, and integrate new
interaction technologies. Unfortunately, these goals are at odds with data
bandwidth and battery capacities, both of which are scaling slowly. We expect
future phones to deploy opportunistic algorithms that multiplex both time and
space in order to improve performance.
 
The overall heterogeneity of deployed devices and standards is another
challenge limiting device-to-device inter-operability, which itself limits
the ability of the phone to interact with all of the other devices it
encounters. We also discussed the importance of integrating the smartphone
with existing Wi-Fi networks to improve connectivity and network performance.
Suggestions were made that peer-to-peer architectures may be worth revisiting
as a way of improving performance, particularly in the case where
infrastructure is lacking.
 
Another property that is not scaling is human attention. We already pay too
much attention to our smartphones to believe that we have achieved the
invisibility desired by early visions of ubiquitous computing, and this
problem seems destined to worsen as the capabilities of phones and the amount
of collected information increase. In order to better optimize our attention
future smartphones must deploy better interfaces allowing more nuanced
interaction with users, including processing emotional cues. To improve the
interaction between humans and their devices, new algorithms must be
developed enabling behavior-based modeling, computing, and testing. In
addition, user interfaces need to be reconsidered, including a number that,
while unsuitable for larger devices, may be well-suited to phones. The
smartphone of 2020, with its ability to interact seamlessly with objects
around it, will be able to leverage “found” interface elements in the
environment to enable much richer interaction modalities than those possible
on the phone itself.
 
Smartphones hold the potential both to contribute to and to alleviate the
growing data deluge. Large-scale deployment of sensor suites on smartphones
combined with cheap bandwidth and storage will lead to a growing amount of
data produced by the phones of the future. Securing this information---much
of it sensitive and personal---will be one major challenge. 
 
Initially designed as a personal device, smartphones and increasingly
interacting with each other and the environment, creating new opportunities
to steal and misuse information. Developing security and privacy models that
users can understand and adapt to their needs is a critical challenge to the
continued advance of smartphone technology.
 
Interpreting and processing the collected data it is another challenge. There
are opportunities for harnessing the distributed power of large numbers of
smartphones through collaborative computation. These capabilities, if
developed, might complement the continued aggregation of computation in the
cloud. Fundamentally, however, the phone of the future will be a portal to
the intelligent processing and management of data in order to reduce user
distraction and allow them to focus their attention elsewhere.

\section{Recommendations to NSF}

When developing our recommendations to NSF we focused on areas in which the
research community could make significant and distinct contributions.
Industry is already very active in this space and has many advantages,
particularly when working at scale. However, we conclude that there are
opportunities for the academic research community to develop the future
smartphone in directions complementary to those being pursued in industry.

\subsection{Research Support}

We recommend that the NSF develop research programs addressing the key
challenges to realizing the Phone 2020 vision outlined above:

\begin{enumerate}

\item We need to continue the forward development of smartphone and
infrastructural capabilities to support demanding new applications.

\item We must tear down the walls that separate devices from each other and
limit the ability of the phone to fully understand its environment.

\item We need better interfaces allowing the future phone to conserve the
most precious resource of all: human attention.

\item We need phones to help users cope with the ever increasing amount of
data both accessible to them and collected about them.

\item We need security and privacy models that users can understand and adapt
to match their expectations and the current context – the highly dynamic pool
of surrounding devices and communications channels, the social setting, and
the user’s activity. 

\item We believe it is important to understand and document our continued
co-evolution with our mobile devices: how we are not only changing them, but
how they are changing us.

\end{enumerate}

\subsection{Infrastructure Support}

If academics are to succeed in complementing industry, the NSF must provide
them with the resources and infrastructure allowing us to experiment at
scale. NSF can also take a role in partnering with industrial players in
order to gain access to resources like large numbers of phones, air time,
call logs or other large data sets. Further partnerships with industry might
also allow us to do citizen-driven science in other areas leveraging the
smart phones as a pervasive computing platform.

Application distribution channels like the Android Market and Google AppStore
also provide academics with the opportunity to deploy research systems at
scale by leveraging channels established by industry. We can release our own
code on the AppStore, perhaps piggybacking on top of other popular
applications. Users worldwide might be willing to participate in a
large-scale virtual laboratory. At sufficient scale such a laboratory could
provide built-in guarantees to researchers allowing academic research to
reach large numbers of deployed smartphones.

\subsection{Educational Opportunities}

One germane direction for the academic community to explore is in the use of
pervasive smartphone technology to enhance education and learning. The future
phone may enter the classroom and help put lessons in context, as well as
extending the reach of learning outside the classroom in novel ways.
 
We also believe future growth in the smartphone area will rely on educating
the next generation of computer scientists on smartphone development. Of
particular interest are training engineers that can help integrate the phone
and the cloud in ways that harness the properties and capabilities of both.
We recommend support for the continued development of classes based on
smartphone platforms and also those that explore smartphone-cloud
interaction.
